\section{Evidence} %% stories from Nabokov, analysis, what the ever
\label{evidence}

Taking this insight\textemdash that memory is both the location of the greatest experiences and readily accessible\textemdash as an axiom, we proceed to observe how Nabokov's characters leverage it.
We see that they consistently place great emphasis on memories, and that this bias obscures their view of the present. 
Thus old friends are judged by the memory they left behind regardless of whether they have or have not changed in the meantime.
This technique may lead to positive or negative results, but the real curiosity is how frequently it is used.

\vspace{1cm}
\subsection{Returning Characters}
To begin, we review a few of the ways in which protagonists from Nabokov's novels reappear in other works.
We see that characters reappear throughout his literary world as though it were a real community.
Interestingly, these old friends look almost exactly as they did when we last parted, raising suspicion about whether we are seeing them as they are or operating under some bias.
%% We wonder, then, whether our visitors are accurate reflections of how the character is faring today or rather resurgent memories, echoing a past where they enjoyed center stage.

The first example of returning characters comes in \emph{The Luhzin Defense}, Nabokov's third novel. 
Here we meet for a second time Mary and Alfyorov, the title character of Nabokov's first book and her husband.
Alfyorov is the same talkative fool and Mary is still the silent angel.
As Luhzin's wife remarks: ``there is something a little mean and a little pathetic about [Alfyorov]''~\cite[The Defense, 152]{defense}.
On the other hand, Mary is just stunning.
``What a darling, what eyes,'' Mrs. Luhzin waxes~\cite[The Defense, 152]{defense}.
Apparently, years together have not softened Alfyorov nor hardened Mary. 
Indeed, Mary seems to have the same heavenly aura as she did in Ganin's youth. 
Curious, but not altogether impossible.

Next, we have the return of the Dreyers in \emph{Laughter in the Dark}. 
At the end of \emph{King, Queen, Knave}, Mrs. Dreyer had passed away yet she returns to send an lunch invitation to Albinus and his wife just as Albinus's illicit relationship with Margot becomes public~\cite[86]{nabokov1960laughter}.
Perhaps they are alive and well, as the reader best remembers them rather than how they actually wound up.
Perhaps it is only the ghost of Mrs. Dreyer, trying to save Albinus's family from ruin.
Either way, the Dreyers make a return in a moment crucial to the happiness of another couple.

Numerous other characters make appearances in later novels, and here the examples number too many to count. 
Hermann appears, evil as sin, or sorry Humbert wanders on stage, let out for his yearly walk~\cite[xiii]{nabokov1989despair}.
Plus, later characters are mixes of previous ones, like Cincinnatus' wife Marfin'ka who is reminiscent of Mary.
Still, we never see a character change between appearances. 
Not once does a character return in a new setting, or with new prospects or values.
They forever remain as they were in their debut into Nabokov's literary world.
On one hand, one might argue that this is done so that the characters are recognizable.
There would be little point to a visit from, say, Pierre Delalande if he were drastically changed from his perfect, outrageous self.
Still, this practice denotes an inability to change and an inability to let go of the past. 
The lack of change is evident by the circumstances of each reappearance and the inability to let go is demonstrated by the fact that these characters appear at all in a book they play no obvious role in.

\vspace{1cm}
\subsection{Case Analysis}
Although thought-provoking, the fact that reappearing characters remain unchanged is not quite convincing that people in Nabokov's world remain static.
Thus, to further illustrate the point that characters use memory (oftentimes deliberately) as an alternative to the present, even to the point that the present is completely ignored or misunderstood, we examine cases and characters within the novels they appear.
In their respective books, these are the first-person characters\textemdash those to whom the rest of the world is relative. 
When we claimed within the introduction that the world never changes, only the first-person presence, we referred to these characters and their perception of their surroundings.

\vspace{1cm}
\subsubsection{\emph{Look at the Harlequins}}
As a motivating example, we take the Duke of Ferrara from \emph{Look at the Harlequins}.
He owns a treasured painting of his deceased wife; however, he ``admires the beautiful portrait of \ldots much more than he had appreciated the original,'' that is, his then-living wife~\cite[3]{toker}.
This shows that the Duke prefers the image, idea, or memory of his wife.
She is not (or rather, was not) much value to him as a living person in her own right. 
Rather her value was purely aesthetic and static. 
For what is a painting but a privileged memory\textemdash an intricately detailed selection from the past~\cite{sartre1964nausea}. 
Indeed, the Duke viewed her as a painting before she was even dedicated to the page. 
%% No wonder, then that he liked the work so much. It was what he'd been dreaming for!

Of note is that \emph{Look at the Harlequins} was Nabokov's last novel. 
At this point in his life, the author was committed to and sure of his ideals. 
That the Duke was more interested in the painting than his wife was no experiment or idiosyncracy.
Every image in Nabokov's work has an underlying meaning, often meanings wrapped within meanings.
The detail is important; the Duke loves the painting because Nabokov saw this infatuation with the past (or in this case, images of the past) as a common, important theme and deliberately reinforced it.

%% ``deprived the lady of her young life and, so to say, turned her into a picture that he commissioned''~\cite{toker}.

\vspace{1cm}
\subsection{\emph{Lolita}}
Lolita is the classic, most well-known example of a character whose understanding of the world is skewed by his preoccupation with the past.
He is haunted by a childhood lover. 
She appears to him everywhere\textemdash on playgrounds, within schoolhouses, at the malt shop or at the church picnic.
Indeed, wherever little girls might be found there is likely to be one manifestation of a nymphet.
What is a nymphet? 
Humbert describes them as exceptional, sexual little girls with a fantastic power over men of ``infinite melancholy, with a bubble of hot poison in their loins'' like himself~\cite{lolita}.
Really though, they are simply manifestations he sees of his lost Annabel, and have little to do with the girl involved.
Humbert's memories cloud and direct his thinking.

In our first encounter with Humbert, he tells the story of this phantom, his Annabel. 
They met and fell in love one summer by the shoreline, but we separated immediately afterward, never to meet again because Annabel grew ill and died~\cite{lolita}.
Humbert speaks of her with intense passion.
Recounting the night they snuck out and nearly consumated their relationship (their failure is his biggest regret), Humbert trembles: 
\begin{quotation}
``Her legs, her lovely live legs, were not too close together, and when my hand located what it sought a dreamy and eerie expression, half-pleasure, half-pain, came over those childish features''~\cite[7]{lolita}
\end{quotation}
She was no doubt a key memory, the best he ever experienced, so naturally he sought to re-create her. 
It was within his reach to reconstruct the past, so why not strive for that goal.

Lolita, the little girl who falls prey to Humbert, suffers greatly from his misinterpretations of her as Annabel.
He sees in Lolita the reincarnation of his dear one. 
And perhaps, in some ways, he is right. 
It is quite possible that Lo has the same eyes, laugh, charm, or habits. 
However, Humbert assumes Lolita is as mature as Annabel would be, had she survived. 
He assumes Annabel's spirit lives on inside Lolita, which is altogether false. 
Thus when he claims the 12-year-old Lolita, he believes himself part of a reunion, not a rape~\cite{lolita}.
This erraneous premise stays with him for years, until the end of the novel, when he realizes that he has ruined Lolita's life.
``The hopelessly poignant thing,'' he ultimately realizes, is ``the abscence of her voice'' from the happy playground~\cite[223]{lolita}.
At any rate, we see that Humbert was so enraputed by his desire to relive his memories of Annabel that he completely misperceived the present, and to ill-effect.

\vspace{1cm}
\subsection{\emph{Pnin}}
%% TODO watch out for repititions

%% boxing would help him.
%% he does not box, and we explain why (after we've shown how it would help)
\emph{Pnin} offers an example of how preserving loved ones as discrete memories offers comfort.
Whereas the illusion arrives and remains effortlessly for other characters, Pnin of actively fights reality.
Yet he is not delusional or otherwise mentally ill, rather he does so to protect his sanity.
Were he more successful about preserving an old memory and ignoring the hereafter, he would lead a happier life.

The tale of the lovable, bumbling Professor Timofey Pnin wholly consists of little episodes.
Many of these are told in retrospect, for example when Pnin recounts his dear wife Liza (before she left him for another man) or his emigration to America~\cite{pnin}.
Upon reaching chapter 5, the reader is well-acquainted with the complications and abnormalities surrounding Pnin's life.
Still, his recollection of Mira, an old sweetheart, arrives poignantly at the chapter's end.
Pnin provides a wisp of their fond memories together, tells of how they were separated, and, quite abruptly, informs the reader that she was killed in a Nazi concentration camp.

%% ``Only in detachment of an incurable complaint, in the sanity of near death, could one cope with this for a moment.
%% In order to exist rationally, Pnin had taught himself, during the last 10 years, never to remember Mira Belochkin''~\cite[99]{pnin}.
Now we can better understand why he tells us, in the previous paragraph, that ``in order to exist rationally, Pnin had taught himself, during the last 10 years, never to remember Mira Belochkin''~\cite[99]{pnin}.
If Pnin could only keep her image as the young, vibrant girl he first met, whose ``gentle heart one had heard beating in the under one's lips in the dusk of the past,'' he can remain happy, comforted that this same spirit\textemdash the Mira he loved\textemdash is still alive and enliviening the world somewhere, in some foreign corner~\cite[100]{pnin}.
However Pnin is too plagued by his own misfortunes to avoid recalling Mira's fate.
He cannot assure himself that the Mira he loved is loving elsewhere because he knows too well that she is finished, never again to laugh, cry, or get angry, and the thought terrifies him.

Moreover the exact manner of her death was never recorded, only that she was buried in the concentration camp.
Pnin thusly imagines a frightening number of ends for his darling, ranging from ``an injection of phenol to the heart'' to immolation on ``a gasoline-soaked pile of beechwood''~\cite[100]{pnin}.
%% ``a sham shower bath with prussic acid''
%% ``inoculated with filth, tetanus bacilli, broken glass''
Again and again Mira dies in his mind; sweet, youthful Mira, ``too weak to work though still smiling,'' too delicate and loving to ever harm another living thing. 
These imaginations creep in to his memories, driving Pnin to the conclusion that ``no conscience, and hence no consciousness, could be expected to subsist in a world where Mira's death were possible''~\cite[100]{pnin}. 
Death-obsessed Pnin cannot accept her end as inevitable, nor can he forget it.
If her memory would stay put, unaffected by what happened after she and Pnin parted, our hero would smile rather than shrink away at mention of her name.

He tries, too, to stop remembering Mira after enjoying the brief recollection of her ``among tall tobacco flowers'', rising from his seat and ``walking away from the house''~\cite[99]{pnin}.
Indeed, Pnin is a Nabokovian anomoly in that a number of his memories are affected by the future.
His wife Liza is always recalled fondly despite her past and future transgressions, and it stands to reason that Victor will remain a charming, lanky boy and Dr. Falternfels will remain a poltergiest, but others like his parents and Mira are remembered by their deaths.
However, the poor man seems doomed to disappointment. 
Mira is taken from him, Liza abandons him (returning only to scrape the dregs of his wallet), and by the end of the novel he has lost his job and home.
%% It seems, however, that Pnin is like any other character in Nabokov's canon in that other people remain to him exactly as he left them.
%% In the case of Mira, as with his father, Dr. Pavel Pnin, and his murdered friend Vanya Bednyashkin, they remain as they were last brought to his attention\textemdash dead~\cite[18]{pnin}.
In this way, his unusually dynamic memories may be understood as a trained reaction.
His mind jumps to the morbid because experience has taught the sad man to expect the worst of life.
%% %% Liza, etc, who remain in a different way
%% This is in contrast to his memories of 
%% end, 

\vspace{1cm}
\subsection{\emph{Mary}}
%% transition from Mira
Pnin's Mira is strangely reminiscent of \emph{Mary}, Queen of Memory, and as such provides a nice segue to begin talking about the familiar apparition.
Mary, Mashenka, Marfin'ka, Tamara\textemdash whatever she be called\textemdash is a powerful, recurring force in Nabokov's work.
His first love in life and a powerful reminder of his days in Russia, Mary never changes to her many beholders.

%% Mary the novel
Nabokov's first novel was centered entirely and explicitly around recollections of dear Mary.\footnote{Exactly who ``dear Mary'' belongs to is up for debate. For now let us say she is Ganin's.}
The protagonist, Ganin, is wasting away in some emigr\`e hostel when he learns that his fellow boarder, Alfyorov, will be importing his wife from Russia.
She is none other than Ganin's Mary, the sweetheart of his youth, and he spends the rest of the week in a fog happily recollecting his time with her.
Slowly, Ganin develops a plan in which he and Mary will run away together. 
Yes, he will steal her from her husband (her husband who does not matter\textemdash how can anyone but Ganin and Mary matter?) and leave the dirty hostel for a brilliant and new life together.
Just as they had been happy in youth so too will they be happy together now.

Ganin continues remembering and loving the memory up until the moment of Mary's arrival.
At which point, he comes to the subtle realization that ``other than that image,'' his recollection, ``no Mary existed, nor could exist''~\cite[135]{mary}.
%% Not sure how to elicidate or clarify beyond that. Think I need a new paragraph

%% scholarly article
Scholars debate about what epiphany Ganin realized at the end of the novel.
One interpretation is that throughout the book, Ganin had been on a journey of uncovering deception and facing the truth in his life.
It began when he rejected the facade of his girlfriend Lyudmila and continued as he saw through the ugliness beneath the ballerinas' makeup.
However his infatuation with Mary's image represented another challenge, and it was only when he realized he too had been creating an illusion that Ganin was able to leave Berlin, his purgatory~\cite[61]{laursen1996memory}.
%% Where am I going? Gonna dispute this? It'd be nice if it could be an ally

But to see past illusions like this, particularly to dispel all illusions, is a task for God, not man. 
No matter how insightful or careful Ganin or any other human may be, they cannot clear themselves of illusions.
Instead, Ganin realized that he had within him the one, true Mary. 
He understood, as the last lines of the novel so clearly read, that the body looking like Mary and married to Alfyorov had no connection at all to his darling.
``No matter who Alfyorov's wife may turn out to be, she is not the Mary whom Ganin had loved''~\cite{toker}.
Thus he was able to calmly leave Berlin because he knew that the most valuable possession, memory, was already his.
That other thing was worthless, so it was no trouble at all to leave it at the train station.

%% %% mary the real
%% %% TODO consult Mr. Boyd
%% She appeared but once in Nabokov's future, and here include the passage in which the author described her in \emph{Speak, Memory}.
%% %% TODO get passage from SPEAK

%% ``During the beginning of that. summer and all thorough the previous one, Tamara's name had kept- cropping up (with the feigned naivete so typical of Fate, when meaning business) here and there on our estate (Entry Forbidden) and on my uncle's land (Entry Strictly Forbidden) on the opposite bank of the Oredezh. I would find it written with a stick on the reddish sand of a park avenue, or penciled on a whitewashed wicket, or freshly carved (but not completed) in the wood of some ancient bench, as if Mother Nature were giving me mysterious advance notices of Tamara's existence. That hushed July afternoon, when I discovered her standing quite still (only her eyes were moving) in a birch grove, she seemed to have been spontaneously generated there, among those watchful trees, with the silent completeness of a mythological manifestation''~\cite[229-230]{speakmemory}.



%% We also have characters reappeearing, and need a place to talk about that. Maybe after mary can go KingQueenKnave, which has a visit
%% or was it Laughter in the Dark?

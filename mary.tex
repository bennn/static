\subsection{\emph{Mary}}
%% transition from Mira
Pnin's Mira is strangely reminiscent of \emph{Mary}, Queen of Memory, and as such provides a nice segue to begin talking about the familiar apparition.
Mary, Mashenka, Marfin'ka, Tamara\textemdash whatever she be called\textemdash is a powerful, recurring force in Nabokov's work.
His first love in life and a powerful reminder of his days in Russia, Mary never changes to her many beholders.

%% Mary the novel
Nabokov's first novel was centered entirely and explicitly around recollections of dear Mary.\footnote{Exactly who ``dear Mary'' belongs to is up for debate. For now let us say she is Ganin's.}
The protagonist, Ganin, is wasting away in some emigr\`e hostel when he learns that his fellow boarder, Alfyorov, will be importing his wife from Russia.
She is none other than Ganin's Mary, the sweetheart of his youth, and he spends the rest of the week in a fog happily recollecting his time with her.
Slowly, Ganin develops a plan in which he and Mary will run away together. 
Yes, he will steal her from her husband (her husband who does not matter\textemdash how can anyone but Ganin and Mary matter?) and leave the dirty hostel for a brilliant and new life together.
Just as they had been happy in youth so too will they be happy together now.

Ganin continues remembering and loving the memory up until the moment of Mary's arrival.
At which point, he comes to the subtle realization that ``other than that image,'' his recollection, ``no Mary existed, nor could exist''~\cite[135]{mary}.
%% Not sure how to elicidate or clarify beyond that. Think I need a new paragraph

%% scholarly article
Scholars debate about what epiphany Ganin realized at the end of the novel.
One interpretation is that throughout the book, Ganin had been on a journey of uncovering deception and facing the truth in his life.
It began when he rejected the facade of his girlfriend Lyudmila and continued as he saw through the ugliness beneath the ballerinas' makeup.
However his infatuation with Mary's image represented another challenge, and it was only when he realized he too had been creating an illusion that Ganin was able to leave Berlin, his purgatory~\cite[61]{laursen1996memory}.
%% Where am I going? Gonna dispute this? It'd be nice if it could be an ally

But to see past illusions like this, particularly to dispel all illusions, is a task for God, not man. 
No matter how insightful or careful Ganin or any other human may be, they cannot clear themselves of illusions.
Instead, Ganin realized that he had within him the one, true Mary. 
He understood, as the last lines of the novel so clearly read, that the body looking like Mary and married to Alfyorov had no connection at all to his darling.
``No matter who Alfyorov's wife may turn out to be, she is not the Mary whom Ganin had loved''~\cite{toker}.
Thus he was able to calmly leave Berlin because he knew that the most valuable possession, memory, was already his.
That other thing was worthless, so it was no trouble at all to leave it at the train station.

%% %% mary the real
%% %% TODO consult Mr. Boyd
%% She appeared but once in Nabokov's future, and here include the passage in which the author described her in \emph{Speak, Memory}.
%% %% TODO get passage from SPEAK

%% ``During the beginning of that. summer and all thorough the previous one, Tamara's name had kept- cropping up (with the feigned naivete so typical of Fate, when meaning business) here and there on our estate (Entry Forbidden) and on my uncle's land (Entry Strictly Forbidden) on the opposite bank of the Oredezh. I would find it written with a stick on the reddish sand of a park avenue, or penciled on a whitewashed wicket, or freshly carved (but not completed) in the wood of some ancient bench, as if Mother Nature were giving me mysterious advance notices of Tamara's existence. That hushed July afternoon, when I discovered her standing quite still (only her eyes were moving) in a birch grove, she seemed to have been spontaneously generated there, among those watchful trees, with the silent completeness of a mythological manifestation''~\cite[229-230]{speakmemory}.


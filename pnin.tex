\subsection{\emph{Pnin}}
%% TODO watch out for repititions

%% boxing would help him.
%% he does not box, and we explain why (after we've shown how it would help)
\emph{Pnin} offers an example of how preserving loved ones as discrete memories offers comfort.
Whereas the illusion arrives and remains effortlessly for other characters, Pnin of actively fights reality.
Yet he is not delusional or otherwise mentally ill, rather he does so to protect his sanity.
Were he more successful about preserving an old memory and ignoring the hereafter, he would lead a happier life.

The tale of the lovable, bumbling Professor Timofey Pnin wholly consists of little episodes.
Many of these are told in retrospect, for example when Pnin recounts his dear wife Liza (before she left him for another man) or his emigration to America~\cite{pnin}.
Upon reaching chapter 5, the reader is well-acquainted with the complications and abnormalities surrounding Pnin's life.
Still, his recollection of Mira, an old sweetheart, arrives poignantly at the chapter's end.
Pnin provides a wisp of their fond memories together, tells of how they were separated, and, quite abruptly, informs the reader that she was killed in a Nazi concentration camp.

%% ``Only in detachment of an incurable complaint, in the sanity of near death, could one cope with this for a moment.
%% In order to exist rationally, Pnin had taught himself, during the last 10 years, never to remember Mira Belochkin''~\cite[99]{pnin}.
Now we can better understand why he tells us, in the previous paragraph, that ``in order to exist rationally, Pnin had taught himself, during the last 10 years, never to remember Mira Belochkin''~\cite[99]{pnin}.
If Pnin could only keep her image as the young, vibrant girl he first met, whose ``gentle heart one had heard beating in the under one's lips in the dusk of the past,'' he can remain happy, comforted that this same spirit\textemdash the Mira he loved\textemdash is still alive and enliviening the world somewhere, in some foreign corner~\cite[100]{pnin}.
However Pnin is too plagued by his own misfortunes to avoid recalling Mira's fate.
He cannot assure himself that the Mira he loved is loving elsewhere because he knows too well that she is finished, never again to laugh, cry, or get angry, and the thought terrifies him.

Moreover the exact manner of her death was never recorded, only that she was buried in the concentration camp.
Pnin thusly imagines a frightening number of ends for his darling, ranging from ``an injection of phenol to the heart'' to immolation on ``a gasoline-soaked pile of beechwood''~\cite[100]{pnin}.
%% ``a sham shower bath with prussic acid''
%% ``inoculated with filth, tetanus bacilli, broken glass''
Again and again Mira dies in his mind; sweet, youthful Mira, ``too weak to work though still smiling,'' too delicate and loving to ever harm another living thing. 
These imaginations creep in to his memories, driving Pnin to the conclusion that ``no conscience, and hence no consciousness, could be expected to subsist in a world where Mira's death were possible''~\cite[100]{pnin}. 
Death-obsessed Pnin cannot accept her end as inevitable, nor can he forget it.
If her memory would stay put, unaffected by what happened after she and Pnin parted, our hero would smile rather than shrink away at mention of her name.

He tries, too, to stop remembering Mira after enjoying the brief recollection of her ``among tall tobacco flowers'', rising from his seat and ``walking away from the house''~\cite[99]{pnin}.
Indeed, Pnin is a Nabokovian anomoly in that a number of his memories are affected by the future.
His wife Liza is always recalled fondly despite her past and future transgressions, and it stands to reason that Victor will remain a charming, lanky boy and Dr. Falternfels will remain a poltergiest, but others like his parents and Mira are remembered by their deaths.
However, the poor man seems doomed to disappointment. 
Mira is taken from him, Liza abandons him (returning only to scrape the dregs of his wallet), and by the end of the novel he has lost his job and home.
%% It seems, however, that Pnin is like any other character in Nabokov's canon in that other people remain to him exactly as he left them.
%% In the case of Mira, as with his father, Dr. Pavel Pnin, and his murdered friend Vanya Bednyashkin, they remain as they were last brought to his attention\textemdash dead~\cite[18]{pnin}.
In this way, his unusually dynamic memories may be understood as a trained reaction.
His mind jumps to the morbid because experience has taught the sad man to expect the worst of life.
%% %% Liza, etc, who remain in a different way
%% This is in contrast to his memories of 
%% end, 

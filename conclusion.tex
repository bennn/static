\section{Conclusion}
\label{conclusion}
%% TODO Summarize contribution
%% This might have been on essay on the importance of memory in the works of Nabokov, but for the fact that each recollection is tinted in some way, twisted towards a common theme. 
%% Memories are subjective. 
%% Experiences are subjective; the whole of life is subjective, and this is brutely manifest in Nabokov's works.
Thus we have seen that memory plays a tremendous role in how Nabokov's characters interface with the world around them. 
Not only does it serve as a positive reminder of good times and inspiration to carry on, it actually changes (or rather, keeps from changing) their interpretation of the world.
Characters use memory to shape their perception to the extreme that other people become more like static images, reincarnations of an idealized past.
The way they reappear unchanged in future works throughout Nabokov's canon only affirms this hypothesis.

We have seen that the effects of this philosophy can be evil, as was the case for Humbert, useful, as was for Pnin and Ganin, or curious, as was for the Duke of Ferrara.
Furthermore, we saw examples indicating that this notion of relativistic experience has a basis in other literary works. 
We might have gone ever deeper, comparing Nabokov's memory-intensive worldview with Stoicism and the general relativistic experience with Aristotalean ethics but for space considerations.
Regardless, the use of memory to mould one's life experiences is a fascinating, effective tool of Nabokov's and well worth close inspection.\footnote{And quite possibly adoption, but we leave that question as open-ended.}

%% Future hopes and dreams
One topic which have have elided thus far is the formation of memories.
We take the creation of the canonical, stone-carving memory as a given and proceed from there. 
Indeed, this was a crucial first step towards establishing our argument, that Nabokov kept his characters like butterflies in glass and assumed the rest of the world operated likewise.
However some memories are more likely candidates to stick than others as the reference.
Despite the randomness of experience and uncontrollable subconscious, we can identify traits common among key memories.
These, we imagine, are quite important to fully appreciating Nabokov's sublime craft, but we leave this analysis for a future endeavor.

%% Audacity
As the narrator of \emph{Pnin} remarks early on in the novel, ``one of the main characteristics of life is discreteness''~\cite[12]{pnin}.
%% ``Unless a film of flesh envelops us, we die. Man exists only insofar as he is separated from his surroundings. The cranium is a space-traveler's helmet. Stay inside or you perish. Death is divestment, death is communion. It may be wonderful to mix with the landscape, but to do so is the end of the tender ego''~\cite[12]{pnin}.
``Man exists only insofar as he is separated from his surroundings. The cranium is a space-traveler's helmet. Stay inside or you perish''~\cite[12]{pnin}.
Whether or not this is sound advice, it is certainly telling of Nabokov's philosophy.
Here is the axiom summarizing and explaining all his characters' actions and behaviors, the fundamental assumption they all evidently carry.
Experiences are subjective, memories are subjective, indeed the whole of life is subjective.
My observations are completely different from yours, and at the end of the day my experiences are the only ones that persist and truly matter.
I remember you, fondly or otherwise, as light reflected through my spaceman's visor.
That image is how you will remain as I grow up and grow old, as I change and see new things.
This both comforts and protects me.
As I wander through a strange an alien world, I have my personal photo album, my collection of faces and recollections that will never change, never fade or wrinkle at the edges.
And so I have the strength to march forward, knowing at least that my past is secure.
%% Not sure where to go next, where to end

%% IT AIN'T FINISHED YET


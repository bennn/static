\section{Background}
\label{background}
%% For those unfamiliar?
%% Relevant bibliography?
%% Russian emigre, prolific and adored author, genius writer incapable speaker

%% memory matters
%% TODO validate

%% time matters
While memory is an essential theme in Nabokov's works, of similar or greater importance are Nabokov's views on time. 
Indeed, his detailed writings on the subject and manner of time have caused some circles to herald Nabokov as a serious author of science fiction~\cite{swanson}.
Regardless of what the author himself might have thought of this praise, that he received such praise at all is certainly telling of his thought-provoking and deep works.

For Nabokov, time is not a directed or continuous thing; it is nothing like the river of poetry or fantasy~\cite{time}.
Time is not orderly or self-contained. 
Rather, time is more like a vortex or an infinitely tangled web\textemdash a wild ocean of past and future.
It is a congealed mess of loose ends, discontinuities, and unexpected connections, a Gordian knot of paths and interconnections.
Yet these descriptions ultimately fail.
They can only direct our intuitions away from the incorrect, not convey the true nature of time.
Time is something beyond human understanding.
Indeed, to quote from \emph{Ada}'s Van Veen, ``We can never know Time. Our senses are simply not meant to perceive it''~\cite{ada}.
Time is the work of God, thus the best efforts of man to describe it are doomed to fail.
What matters for our present understanding is that iteractions with the past and future are not only possible, but likely. 
The whole of time is connected and our perception of it is fundamentally relative. 
%% We must constantly be aware that there is more to time than what we can espy on the surface.
Although the mysteries of time remain hidden to us who are caught in its web, we must be aware that the whole network of time is connected and both past and future are both with us in some manner at present.

Discussions of Nobkov's philosophy concerning time appear ubiquitously in his work, but none more strongly than in \emph{Ada}.
The novel itself spans over 100 years and is summarized by Nabokov scholar Alfred Appel Jr. as ``a philosophical investigation into the nature of time''~\cite{appelada}.
Much of the text explores the significance and properties of time; even the protagonist, Doctor Van Veen, focuses his creative efforts on a book called ``The Texture of Time'' and it is through this medium that a number of Nabokov's views on time become apparent.
%% Understanding these provides insight crucial for reading his other works.
Here we are formally instructed on the dynamic nature of time, the way past and future mingle incestuously, and the utter incomprehensibility of time to humans.
Moreover, we learn of Nabokov's opinion on the present, thus gaining crucial insight as to why memory is so important.
Van Veen, in writing ``The Texture of Time'', states that ``we can not enjoy the true present, which is an instant of zero duration''~\cite{appelada}.
Rather we must derive pleasure from warm memories of the past and hopes and dreams of the future. 
These valuables, unlike the transitory present, are what we can truly hold on to.
In particular, we can claim the past.
Futures may be altered and the present moment leaves us before we are aware of its arrival, but the past is never-changing. 
The past is comforting, the past can amuse and elighten. 
As such, we can see why the past has such a prominent role in Nabokov's work.

Additional insight follows from Nabokov's background. 
After living a charmed childhood on his family's estate in Russia, Nabokov's family lost their wealth and prestige during the Revolution and Vladimir spent the rest of his life in exile~\cite{boyd1993vladimir}.
Born into ``stupendous wealth'' and promptly thrust into poverty and exile, it is little wonder that Nabokov holds memories of the past especially dear~\cite[Boyd, 3]{boyd1993vladimir}.
One can imagine these memories kept him alive and motivated during his time as an unknown emigr\`e writer in Berlin or as a young husband searching desperately for employment, especially when you read his recollections of Russia from \emph{Speak, Memory}:
\begin{quotation}
``Tamara, Russia, the wildwood grading into old gardens, my northern birches and firs, the sight of my mother getting down on her hands and knees to kiss the earth every time we came back to the country from town for the summer, et la montagne et le grand chene\textemdash these are things that fate one day bundled up pell-mell and tossed into the sea, completely severing me from my boyhood''~\cite[249-250]{speakmemory}.
\end{quotation}


%% they both mix in all the way and such forth. Static + Memory. Time stops for memory. Memory > time, controls time, stops the forward flow
Now we arrive at a better understanding of how memory factors in to Nabokov's work. 
Memory is valuable in that it may be relied upon never to change or grow bitter, and moreover memory is quite often (at least, according to experience) the site of the sweetest and most precious experiences.
If we also take Nabokov's philosophy of a tightly knit past, present, and future, we now have a tight, ever-present bond with these memories.
No matter where the future takes us, they are accessible. 
Through the mysterious workings of time, we can reach the heaven that lies in our past. 
This conclusion is fundamental for understanding Nabokov's characters.

%% Indeed, it takes precedence over the present and more recent past\textemdash memories can override subsequent changes 
%% This is perfectly in line with Nabokov's 

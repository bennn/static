\section{Introduction}
\label{intro}
%% we tend to think or like to think that people change
%% pervasive in vladimir's work is the idea that people never change
%% characters are always the same, stuck in limbo, never any different no matter how they want to change
%% hell hell hell

%% ideal, pervasive, axiom, belief, fundamental assumption, 
Pervasive throughout Nabokov's work is a conviction that people, represented as characters in the novel, never change. 
The sole exception is the author himself, who learns and grows from his experiences.
Those around him are static.
Life passes by and tragedies shake our world but the people who have been remain unchanged by the tumult, engaging in the same activities, holding true to the same beliefs, and showing the same characteristics.

We identify this relativistic, self-centered bias in Nabokov's work and then explore reasons for why it exists.
In particular, we survey \todo{these works}, noting the common patterns followed by the protagonists and the occasions where characters from previous novels reappear.
Where one might expect a character to change or develop, they consistently remain as they are. 

This philosophy, which we brand \emph{\philo}, is a product of Nabokov's personality and experiences, reflecting his beliefs and hopes concerning the world around him.
%% helps us categorize memories and recover from upsets
A variant of narcissism, \philo~is not specific to Nabokov but rather a commonly held perspective, natural to human sensibilities and useful for contending with a dynamic world.
Like binary thinking, it cruches the world around us into manageable portions.
Taking Nabokov's work as a first example, we introduce this philosophy and motivate its practicality.
Make no mistake, the focus here is on a previously unexplored theme apparent Nabokov's work, this practical application being yet another lesson we may glean from the master.


\section{Introduction}
\label{intro}
%% we tend to think or like to think that people change
%% pervasive in vladimir's work is the idea that people never change
%% characters are always the same, stuck in limbo, never any different no matter how they want to change
%% hell hell hell

Memory is a central, well-established theme in Nabokov's work. 
Hardly a discussion of Nabokov's life or works fails to mention his rapture with memory, the way it motivates and shapes his writing. %%~\cite{yardley,laursen1996memory,shapiro2009sublime,shapiro1998delicate,richardson}.
To quote from the aptly-named autobiography \emph{Speak, Memory}, Nabokov himself remarks that ``the act of vividly recalling a patch of the past is something that I seem to have been performing with the utmost zest all my life''~\cite{speakmemory}.
The landscapes he painted both with the pen and with the brush were characterized by this intense attention to detail and knack for observation, which was only possible though serious treatment and review of the past.

%% ideal, pervasive, axiom, belief, fundamental assumption, 
Less examined, however, is the curious and powerful manner in which memories affect and influence characters.
More than cherished recollections or the source of insight, memories are the overriding motivator for protagonists' actions towards and perceptions of other characters.
Instead of a supplementary influence, memories \emph{are} the influence that determines how characters behave.
As a result, we have this pervasive theme throughout Nabokov's work that people, represented as characters in a novel, never change. 
The sole exception is the author or protagonist himself, who alone learns and grows from his experiences.
Those around him are static.
Canonicalized by memories, they do not budge.
Years go by, life passes by, and tragedies shake our world but the people who have been remain unchanged by the tumult, engaging in the same activities, holding true to the same beliefs, and demonstrating the same characteristics.
This is how we, the first-person, see them, and so they are.
%% The image, once formed by a canonical experience and solidified in memory, is permanent.

After providing additional background and a brief treatment of Nabokov's philosophy of time, we identify this relativistic, self-centered bias in Nabokov's work and then explore reasons supporting it.
First, we observe the instances where characters from past novels reappear in future works and note how infrequently they change.
Where one might expect a character to have grown developed, they have not.
Like sculptures or photographs they remain as they were remembered.
Next, we survey in greater detail the cases where memory dominates characters' behavior and interactions within particular novels.
We note the scenarios and motives where memories dominate a character's perception of the world around them.
These explorations begin with the curious habits of \emph{Look at the Harlequins}' Duke of Ferrara, transition into Humbert's obsessiveness in \emph{Lolita} and the vulnerable nature of \emph{Pnin}, and finally culminate in a review of Ganin in \emph{Mary}, the first and most significant representative.

Taking into account the common threads among Nabokov's works, we arrive at the conclusion that characters' experiences are relative.
To the protagonist\textemdash the first-person\textemdash the world appears in a single, unchanging way. 
Regardless of the true nature of other people, regardless of how they see or present themselves, the first-person views them in one dimension only.
This perception is formed and wholly decided by some outstanding memory and stands thenceforth, thus giving new significance to the power of memory in Nabokov.

%% Stay put like a good little memory

%% This philosophy, which we brand \emph{\philo}, is a product of Nabokov's personality and experiences, reflecting his beliefs and hopes concerning the world around him.
%% %% helps us categorize memories and recover from upsets
%% A variant of stoicism, \philo~is not specific to Nabokov but rather a commonly held perspective, natural to human sensibilities and useful for contending with a dynamic world.
%% Like narcissism in small doses, it buffers a persons self-esteem against uncertainty.
%% Like binary thinking, it crunches the world around us into more manageable portions.
%% Taking Nabokov's work as motivation, we introduce this philosophy and motivate its practicality.
%% That being said, the focus here is decidedly on analyzing a previously unexplored theme apparent in Nabokov's work, \philo~being yet another lesson we may glean from the master.


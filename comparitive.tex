\section{Comparison to other Authors}

Thus we have seen four occasions where characters' memory affects their interpretation of the present and demonstrates the relativity of experience\textemdash how one person's view of the world can drastically vary from the accepted or real.
The Duke of Ferrara preferred a painting\textemdash the soft, well-made, and silent object\textemdash over his wife, Humbert was so entralled by his quest for Annabel that he destroyed the life of a little American girl, Pnin sought refuge in memory to contend with the hard realities of life, and Ganin realized happiness though the past, that his lost love was not lost at all but part of his present and future.
However, Nabokov is not the only author to highlight the relative nature of memory.
Others have used advanced similar points of view. 
%% Nabokov is by far the most insistent, but their passages are telling of the subjective nature of life and experience.
Here we gives two brief examples, taken from \emph{All the King's Men} and \emph{Dandelion Wine}, to further reinforce the fact that experience is relative.
Moreover, this analysis leads us to believe that Nabokov's tool of using memory to shape perception is the symptom of a more broadly used technique by which humans process sensory information quickly by categorizing other people and refraining from changing category labels or membership.

\vspace{1cm}
\subsection{\emph{All The King's Men}}
Jack Burden, protagonist of Robert Penn Warren's \emph{All The King's Men}, remarks as he begins a long lonely journey, down a rainy highway after visiting his mother, that ``you are not you except in terms of relation to other people''~\cite{warren1982all}
To Jack, this comes as a comfort.
He looks forward to ``really get[ting] some rest'' as he drives from ``the you [he] just left'' to ``the you [he] will be'' at his destination~\cite[88]{warren1982all}.

Whether or not the prospect of being ``nobody'' for a while is exciting, we find that Jack is acutely aware that other persons view him in many different ways. 
Though he is but one person, one body and one flesh, he may be a secretary, child, husband, or friend depending on the situation. %%Jackie Bird
%% where the hell is this going? I am a little tired, but I don't think this belongs in a conclusion
He has no one self, no center. 
His entire experience is relative to those around him, based in other people. 
Whether this is a healthy perspective or otherwise, it this philosophy does carry Jack through the larger part of the novel and through many difficult situations.

\vspace{1cm}
\subsection{\emph{Dandelion Wine}}
Ray Bradbury's summer novel \emph{Dandelion Wine} is full of charming vignettes and warm memories.
A few, however, are more sinister.
Among these we have the story of Mrs. Bentley, 72-years old, whose home is filled with ``the paraphenalia of years''~\cite[50]{bradbury}.
She keeps record of everything: ``tickets, old theater programs, bits of lace, scarves, rail transfers'', ``a tiny ring [she] wore when [she] was eight'', her husband's ``high silk hat and his cane'', ``a dress in which she had played the mandarin's daughter at 15''\textemdash her house is stuffed with accumulated belongings~\cite[50-56]{bradbury}.
Trinkets kept in an effort to keep something of the past, to keep memories dear and real and present.

All this changes when she meets two young girls, who arrive at her house and plainly inform her that she never was young and never wore ``ribbons and dresses'' and never was pretty~\cite[57]{bradbury}.
At first indignant and shocked, Mrs. Bentley realizes that these girls cannot see her in any other way. 
They are young things who will grow up and change in time, but for now all Mrs. Bentley can see are two insolent children and all the children can see is a cranky old woman.
``Children are children and old women are old women, and nothing in between.''\cite[54]{bradbury}
%% ``You're in the present, you're trappen in a young now or an old now, but there is no other now to be seen.''\cite[56]

She takes this to mean she must live only in the present, for the present is all she can ever have, and indeed this makes a nice finish to the short chapter.
Mrs. Bentley offers her possessions to the girls and burns the rest in a literal sort of housewarming ceremony, and afterwards the reader is permitted to forget her and finish enjoying the story.
However, perhaps the real point was more subtle\textemdash that their is no truth except your own truth, that which you hold within. 
The girls may then remain little in Mrs. Bentley's eyes and to them she can be an old woman, yet simultaneously she can keep hold of her memories of her younger life.
Indeed, as Mrs. Bentley arrived at her conclusion to live in the present she consulted the memory of her dead husband, which rather contradicts her conclusion that only the present matters.
Truth, at least as much as matters, is something we keep privately.
It is static for an individual, for myself alone, but varied across individuals.

%% \subsection{\emph{1984}}
%% Continuing the theme of truth, and what it means for different people, we turn to Orwell's dystopian classic \emph{1984}.
%% Set in a futuristic version of England, the oppressive government keeps the populance under control by means of extreme propaganda and a ruthless police force.
%% They leverage authority in presenting one version of the truth

%% %% 
%% Winston and Julia, the novel's protagonists, attempt to escape into their own private world.


%% %% 
%% However, the all-knowing government monitors and captures the lovers, subjecting them to imprisonment and interrogation.
%% %% Force him to accept a different truth
%% During Winston's torture, the
%% %% battle over self -> rejected his individual perceptions of truth, the fact that experiences differ. 

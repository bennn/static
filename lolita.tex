\subsection{\emph{Lolita}}
Lolita is the classic, most well-known example of a character whose understanding of the world is skewed by his preoccupation with the past.
He is haunted by a childhood lover. 
She appears to him everywhere\textemdash on playgrounds, within schoolhouses, at the malt shop or at the church picnic.
Indeed, wherever little girls might be found there is likely to be one manifestation of a nymphet.
What is a nymphet? 
Humbert describes them as exceptional, sexual little girls with a fantastic power over men of ``infinite melancholy, with a bubble of hot poison in their loins'' like himself~\cite{lolita}.
Really though, they are simply manifestations he sees of his lost Annabel, and have little to do with the girl involved.
Humbert's memories cloud and direct his thinking.

In our first encounter with Humbert, he tells the story of this phantom, his Annabel. 
They met and fell in love one summer by the shoreline, but we separated immediately afterward, never to meet again because Annabel grew ill and died~\cite{lolita}.
Humbert speaks of her with intense passion.
Recounting the night they snuck out and nearly consumated their relationship (their failure is his biggest regret), Humbert trembles: 
\begin{quotation}
``Her legs, her lovely live legs, were not too close together, and when my hand located what it sought a dreamy and eerie expression, half-pleasure, half-pain, came over those childish features''~\cite[7]{lolita}
\end{quotation}
She was no doubt a key memory, the best he ever experienced, so naturally he sought to re-create her. 
It was within his reach to reconstruct the past, so why not strive for that goal.

Lolita, the little girl who falls prey to Humbert, suffers greatly from his misinterpretations of her as Annabel.
He sees in Lolita the reincarnation of his dear one. 
And perhaps, in some ways, he is right. 
It is quite possible that Lo has the same eyes, laugh, charm, or habits. 
However, Humbert assumes Lolita is as mature as Annabel would be, had she survived. 
He assumes Annabel's spirit lives on inside Lolita, which is altogether false. 
Thus when he claims the 12-year-old Lolita, he believes himself part of a reunion, not a rape~\cite{lolita}.
This erraneous premise stays with him for years, until the end of the novel, when he realizes that he has ruined Lolita's life.
``The hopelessly poignant thing,'' he ultimately realizes, is ``the abscence of her voice'' from the happy playground~\cite[223]{lolita}.
At any rate, we see that Humbert was so enraputed by his desire to relive his memories of Annabel that he completely misperceived the present, and to ill-effect.

\subsubsection{\emph{Look at the Harlequins}}
As a motivating example, we take the Duke of Ferrara from \emph{Look at the Harlequins}.
He owns a treasured painting of his deceased wife; however, he ``admires the beautiful portrait of \ldots much more than he had appreciated the original,'' that is, his then-living wife~\cite[3]{toker}.
This shows that the Duke prefers the image, idea, or memory of his wife.
She is not (or rather, was not) much value to him as a living person in her own right. 
Rather her value was purely aesthetic and static. 
For what is a painting but a privileged memory\textemdash an intricately detailed selection from the past~\cite{sartre1964nausea}. 
Indeed, the Duke viewed her as a painting before she was even dedicated to the page. 
%% No wonder, then that he liked the work so much. It was what he'd been dreaming for!

Of note is that \emph{Look at the Harlequins} was Nabokov's last novel. 
At this point in his life, the author was committed to and sure of his ideals. 
That the Duke was more interested in the painting than his wife was no experiment or idiosyncracy.
Every image in Nabokov's work has an underlying meaning, often meanings wrapped within meanings.
The detail is important; the Duke loves the painting because Nabokov saw this infatuation with the past (or in this case, images of the past) as a common, important theme and deliberately reinforced it.

%% ``deprived the lady of her young life and, so to say, turned her into a picture that he commissioned''~\cite{toker}.
